\documentclass{article}
\usepackage{geometry}
\geometry{
    paper=a4paper,
    top=1.5cm,
    bottom=2cm,
    right=3cm,
    left=3cm,
}

\usepackage{fancyhdr}
\fancypagestyle{pdf}{ 
    \fancyhf{} % clear all header and footer fields
    \fancyhead[R]{Practical Internet of Things with Raspberry Pi}
    \fancyfoot[C]{\thepage}
    \setlength{\headheight}{13pt}
}
\pagestyle{pdf}

\title{Weather Station}
\author{Marius \and Leonard}
\date{\today}

\begin{document}
\maketitle
\newpage
\tableofcontents
\newpage

\section{Introduction}
// Idea and Description\\
The project entails the construction of a comprehensive weather station. 
This station is equipped with sensors to measure temperature, humidity, and air pressure. 
The Sense HAT's LED screen displays one of the three data parameters, selectable via a joystick. 
Additionally, the text color on the LED screen dynamically changes based on predefined thresholds, with blue or red indicating values below or above acceptable limits, respectively, and green denoting normal conditions. 
In cases where at least one parameter exceeds the defined threshold for high values, a signal is visually conveyed through the blinking of an LED on the breadboard.
To monitor and analyze the data systematically, the project integrates with ThingSpeak, facilitating data uploads and providing a platform for visualization. 
Furthermore, the system employs MQTT to enable remote monitoring via an app. 
In the event of sensed values surpassing predefined thresholds, an alarm is triggered, notifying the user through the app.

\section{Hardware}
// Which Hardware did we use\\
The project is executed on a Raspberry Pi 400 and incorporates key components, including the Sense HAT, Si7021 temperature and humidity sensor from Adafruit, BMP3xx temperature and pressure sensor (also from Adafruit), and an LED with an accompanying resistor. 
These components are seamlessly integrated and wired on a breadboard, forming the core infrastructure of the weather station system.

\subsection{Hardware connection}
// structure of Hardware\\
// wiring

\section{Software implementation}
// Code plus explaining\\
// in sections

\section{Results}
// what the "user" sees\\
// Screenshots of ThingSpeak and Apps\\
// Picture of SenseHat?

\end{document}